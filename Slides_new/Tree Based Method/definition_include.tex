%\documentclass[noinfoline]{imsart}
%\usepackage{amsmath,amstext,amssymb}
%%\usepackage[top=1.5in, bottom=1.5in, left=1.2in, right=1.2in]{geometry}
%% settings
%%\pubyear{2005}
%%\volume{0}
%%\issue{0}
%%\firstpage{1}
%%\lastpage{8}
%\arxiv{arXiv:0000.0000}

%\startlocaldefs
%\numberwithin{equation}{section}
%\theoremstyle{plain}
%\newtheorem{thm}{Theorem}
%\endlocaldefs
\usepackage{lipsum} 
\usepackage{amsmath}
\usepackage{amssymb}
\usepackage{amsbsy} 
\usepackage{amsthm}
\usepackage{mathrsfs}
\usepackage{eufrak}
\usepackage{mathrsfs}
\usepackage{color}
\usepackage{verbatim}
\usepackage{graphicx}
\usepackage{bm}
\usepackage{enumerate}
\usepackage{epstopdf} 
\usepackage{natbib}
\usepackage{undertilde}
%\RequirePackage[colorlinks,citecolor=blue,urlcolor=blue]{hyperref}
%\usepackage{subfig}
\usepackage[final]{pdfpages}

\usepackage{algorithm}  %@subhajit
\usepackage{algpseudocode} %@subhajit
\usepackage{algorithmicx}     %@subhajit
\usepackage{undertilde}


\newcommand{\sphere}{{\mathbb{S}}}
\newcommand{\R}{\mathbb{R}}
\newcommand{\LatentV}{V}
\newcommand{\NC}{m}
\newcommand{\Priorf}{f_{prior}}
\newcommand{\FWOne}[2]{{{}_{1}\Psi _{1}\left[{\begin{matrix}(\frac{#1}{2},\frac{1}{2})\\(1,0)\end{matrix}};#2\right]} 
}


\newcommand{\HyPriorMu}{\thetabf}
\newcommand{\HyPriorAlpha}{\alpha}
\newcommand{\HyPriorBeta}{\beta}
\newcommand{\HyPriorK}{\zeta}
\newcommand{\Indicator}[1]{\mathbb{I}({#1 })}
\newcommand{\xb}{\bm{x}}
\newcommand{\bx}{\MakeVec{\bm{x}}}
\newcommand{\bX}{\bm{X}}
\newcommand{\by}{\MakeVec{\bm{y}}}
\newcommand{\bZ}{\bm{Z}}
\newcommand{\bF}{\bm{F}}
\newcommand{\btheta}{\MakeVec{{\bm{\theta}}}}
\newcommand{\Bpi}{\MakeVec{\boldsymbol{\pi}}}
\newcommand{\thetabf}{\MakeVec{\boldsymbol{\theta}}}
\newcommand{\Thetabf}{\boldsymbol{\Theta}}
\newcommand{\taubf}{\MakeVec{\boldsymbol{\tau}}}
\newcommand{\Tr}{Tr}


\newcommand{\bM}{\bm{M}}
\newcommand{\bD}{\MakeVec{\bm{D}}}
\newcommand{\bV}{\MakeVec{\bm{V}}}
\newcommand{\loglikmix}{\mathcal{L}_{\bM,\bD,\bV}}
\newcommand{\hypdc}{{}_0F_1\left(\frac{n}{2},\frac{D_c^2}{4}\right)}


\usepackage{xstring}
\usepackage[normalem]{ulem}
\definecolor{ultramarine}{RGB}{38,29,163}
\newcommand\PalDel[1]{{\color{red} {\sout{#1}}}}
\newcommand\Pal[1]{{\color{ultramarine}{#1}}}
\newcommand\PalRp[2]{\PalDel{#1} \Pal{#2}}
\newcommand\PalCmnt[1]{{\color{ultramarine} {[[[***PAL:  #1 ***]]]}}}

\newcommand{\qedwhite}{\hfill \ensuremath{\Box}}
\newcommand{\SpaceD}{\mathcal{S}_p}
\newcommand{\SpaceM}{\widetilde{\mathcal{V}}_{n,p}}
\newcommand{\SpaceV}{\mathcal{V}_{p,p}}
\newcommand{\SpaceF}{\mathbb{R}^{n,p}}
\newcommand{\StiefelS}{\mathcal{V}_{n,p}}
\newcommand{\SpacePi}{\mathbb{S}_{\pi}}
\newcommand{\ML}{{\cal{ML}}}
\newcommand{\ProdSpace}{\boldsymbol{\Theta}}
\newcommand{\ThetaAndPi}{\Xi}
\newcommand{\ClassML}{\mathcal{C}_{\ML}}


\newcommand{\balpha}{\MakeVec{\bm{\alpha}}}
\newcommand{\bbeta}{\MakeVec{\bm{\beta}}}
\newcommand{\bEta}{\bm{\eta}}
\newcommand{\bd}{{\utilde{\bm{d}}}}
\newcommand{\BoEta}{{\utilde{\boldsymbol{\eta}}}}
%\newtheorem{theorem}{Theorem}[section]
%\newtheorem{theorem}{Theorem}
%\newtheorem{lemma}{Lemma}
%\newtheorem{result}{Result}
\newtheorem{defn}{Definition}
\newcommand{\pdv}[2][]{\frac{\partial#1}{\partial#2}}
\newcommand{\pdvtwo}[2][]{\frac{\partial^2#1}{{\partial#2}^2}}


%\newcommand{\mubf}{\boldsymbol{\mu}}
\newcommand{\mubf}{\MakeVec{\mu}}
\newcommand{\sumI}{ \sum_{i=1}^{n}}
\newcommand{\Ybar}{{\overline{Y}}}

\newcommand{\Expectation}[1]{\mathbb{E}{[#1]}}
\newcommand{\priorXzero}{\Psi}
\newcommand{\iMat}{\mathbf{I}_{p}}

% 
% \newtheorem{thm}{Theorem}[section]
% \newtheorem{cor}[thm]{Corollary}
% \newtheorem{lem}[thm]{Lemma}
%\newtheorem{proposition}{Proposition}

%\newtheorem{theorem}{Theorem}[chapter]%To link the theorem to each chapter uncomment the chapter option
%\newtheorem{lemma}{Lemma}%[theorem]% To link each lemma to a theorem uncomment the theorem option
%\newtheorem{corollary}{Corollary}%[theorem]% To link each corollary to a theorem uncomment the theorem option
% to link a corollary to a chapter change the theorem option to chapter
%\newtheorem{definition}{Definition}%[chapter] %the same is true for both definitions and assumptions
\newtheorem{assumption}{Assumption}%[chapter] %
%\newtheorem{proposition}{Proposition}[chapter]
%\newtheorem{fact}{Fact} %%% added by @subho
\newcommand{\StrongNBD}[2]{S_{#1}{#2}}
\newcommand{\bpi} {\boldsymbol{\pi}}
\newcommand{\bphi} {\boldsymbol{\phi}}
\newcommand{\bb}[1]{\boldsymbol{#1}}
% Definitions of handy macros can go here

\newcommand{\normtwo}[1]{{\left\lVert#1\right\rVert}_2}

\newcommand{\dataset}{{\cal D}}
\newcommand{\fracpartial}[2]{\frac{\partial #1}{\partial  #2}}
\newcommand{\Lesbegue}[1]{\mu_{\btheta_{#1},\bpi_{#1}}}
\newcommand{\fthetapi}[1]{f_{\btheta_{#1},\bpi_{#1}}}
% Heading arguments are {volume}{year}{pages}{submitted}{published}{author-full-names}
\newcommand{\doublehat}[1]{%
    \settoheight{\dhatheight}{\ensuremath{\widehat{#1}}}%
    \addtolength{\dhatheight}{-0.35ex}%
    \widehat{\vphantom{\rule{2pt}{\dhatheight}}%
    \smash{\hspace{-0.5mm}\widehat{#1}}}}

\newcommand{\hyp}{{}_0F_1\left(\frac{n}{2},\frac{D^2}{4}\right)}
\newcommand{\hypinline}{{}_0F_1\left({n}/{2},{D^2}/{4}\right)}

\newcommand{\partialhyp}[1]{\frac{\partial}{\partial\,{d_{#1}}}\,\left[\hyp\right]}

\newcommand{\fracProbZ}[1]{\frac{\langle Z_{ic} \rangle \, #1}{\sum_{i=1}^{N} \langle Z_{ic}\rangle  } }
\newcommand{\EmVar}[1]{\widetilde{#1}^{(c)}}

\newcommand{\IMDY}{{\it{CCPD}}}
\newcommand{\JMDY}{{\it{JCPD}}}

\newcommand{\DYlang}{\frac{\exp(\nu\,\bEta^T\bd)}{{\left[{}_0F_1\left(\frac{n}{2},\frac{D^2}{4}\right)\right]}^{\nu}}}

\newcommand{\logDYlang}{\nu\,\bEta^T\bd - \nu\,\log\left({}_0F_1\left(\frac{n}{2},\frac{D^2}{4}\right)\right)}

\newcommand{\lhyp}{\log\left({}_0F_1\left(\frac{n}{2},\frac{D^2}{4}\right)\right)}

%\jmlrheading{1}{2000}{1-48}{4/00}{10/00}{SS \& JH \& AB}

% Short headings should be running head and authors last names

%\ShortHeadings{BDP and cIBP}{SS \& JH \& AB}
%\firstpageno{1}

\newcommand{\diam}[1]{{{#1}^{\ast}}}

%%% coloring option %%%
\definecolor{auburn}{rgb}{0.53, 0.1, 0.5}
\newcommand{\sss}{\color{auburn}}  %%% for Subhajit
\newcommand{\sse}{\color{black}}
\newcommand{\attn}{\color{red}}
\newcommand{\rms}{\color{magenta}}  %%% for Riten
\newcommand{\rme}{\color{black}}
\newcommand{\MLDensity}{f_{\ML}}
\setlength{\parindent}{0cm}
\newcommand{\posterior}

\newcommand{\variableX}{\bd}
\newcommand{\funch}{\mathfrak{h}}
\newcommand{\IndVzero}[1]{\mathbb{I}({X\in \mathcal{V}^{#1}_0})}
\newcommand{\Rnp}{\mathbb{R}^{n \times p}}
\newcommand{\Rpp}{\mathbb{R}^{p \times p}}
\newcommand{\vecnorm}[1]{\lVert #1\rVert}

\newcommand{\etapsiD}{\eta_{\priorXzero}}
\newcommand{\BoEtapsiD}{\BoEta_{\priorXzero}}

\newcommand{\DMp}{\mathcal{D}^{p \times p}}
\newcommand{\Rplus}{\mathbb{R}_{+}}
\newcommand{\prodMeasure}{\Upsilon}

\newcommand{\m}{{\bf m_{\BoEta}}} 
\newcommand{\SetWithMode}{\mathcal{S}}
\newcommand{\SetWithModePrime}{\mathcal{S}}
\newcommand{\TargetComp}{\mathcal{S}^{\star}}

\newcommand{\ConstCondDen}{K_{\nu, \BoEta}} 

\newcommand{\hyparam}[2]{
    \IfEqCase{#1}{
        {M}{\xi^{#2}_c}
        {V}{\gamma^{#2}_c}%
        
    }
  }
\newcommand{\threepartdef}[6]
{
	\left\{
		\begin{array}{lll}
			#1 & \mbox{if } #2 \\
			#3 & \mbox{if } #4 \\
			#5 & \mbox{if } #6
		\end{array}
	\right.
}

\def\bv{\color{blue}}
\def\ev{\color{black}}
\newcommand{\bch}{\bv }
\newcommand{\ech}{\ev\normalsize}
%\newcommand{\MakeVec}[1]{{\utilde{\bf #1}}}
\newcommand \Measure[2][]{%
  \ifstrempty{#1}{
  \IfEqCase{#2}{
        {M}{\mu}%
        {D}{\mu_1}%
        {V}{\mu_2}
        {X}{\mu}
   }  
  }{
  \IfEqCase{#1}{
  {1}{
   \IfEqCase{#2}{
        {M}{d\mu(M)}%
        {D}{d\mu_1(\bd)}%
        {V}{d\mu_2(V)}
        {X}{d\mu(X)}
        {Y}{d\mu(Y)}
        {MDV} {d\mu(M)\; d\mu_1(\bd) \;d\mu_2(V) }
        }
   } 
   {2}{
   \IfEqCase{#2}{
         {M}{d\mu(M^{\ast})}%
        {D}{d\mu_1(\bd^{\ast})}%
        {V}{d\mu_2(V^{\ast})}
        {X}{d\mu(X^{\ast})}
        }
   }
   {3}{
   \IfEqCase{#2}{
         {M}{\mu(dM^{\star})}%
        {D}{\mu_2(d\bd^{\star})}%
        {V}{\mu_1(dV^{\star})}
        {X}{\mu(X^{\star})}
        }
   }   
   
   } 
  }%
}
  \newcommand{\VONF}{\text{VonMisesFisher}}
\newcommand{\MPGalpha}{\alpha}
\newcommand{\MPGnu}{\nu}
\newcommand{\MPG}{MPG }
\newcommand{\ybin}{y^{(\text{bin})}}


%\newcommand{\abs}[1]{\left \vert  #1  \right\vert  }
\usepackage{caption}
\usepackage{subcaption}

%%%%%%%%%%%%%%%%%%%%%%%%%%%
\newcommand{\IEHC}{\text{IEHC}}







\newcommand \Th[1]{%
  \IfEqCase{#1}{
        {1}{ 1^{\text{st}}}%
        {2}{2^{\text{nd}}}%
        {3}{3^{\text{rd}}}%
  }[{#1}^{\text{th}}]
}
  
  
   \newcommand{\augV}{\text{aux}}
  
  
  
  
  \newcommand{\CDE}{\text{PL}}
\newcommand{\CDEsigma}{\sigma}
\newcommand{\CDEepsilon}{\SVepsilon}
\newcommand{\CDEmu}{\mu}
 % \newcommand{\SVepsilon}{\varepsilon}
  \newcommand{\SVepsilon}{\delta}
 \newcommand{\abs}[1]{\left\lvert{#1}\right \rvert }
 
 
\newcommand{\CPDX }{\text{CPDX}}
\newcommand{\CPDXPar}{\vartheta}
\newcommand{\K}{\mathcal{K}}



\newcommand{\lossFunctionOne}[1]{ \left\{ \abs{ ( \abs{#1}-\SVepsilon)}  + ( \abs{#1}-\SVepsilon)\right\} }

\newcommand{\lossFunctionAlt}[1]{ \abs{  #1-\SVepsilon}  + \abs{ #1+\SVepsilon}-2\SVepsilon }

\newcommand{\lossFunctionAltOne}[1]{   \lossFunctionAlt{ \frac{\left(#1\right)}{\sigma}}}

\newcommand{\lossFunction}[1]{ \left\{ \abs{ \left( \frac{\abs{#1}}{\sigma}-\SVepsilon\right)}  + \left( \frac{\abs{#1}}{\sigma}-\SVepsilon\right)\right\} }
\newcommand{  \Likelihood}{\mathcal{L}}
%\newcommand{\Onebf}{\bf 1}
\newcommand{\Onebf}{{\bf \utilde{1}_{n}}}





\newcommand{\InvGamma}{\text{InvGamma}}
\newcommand{\PriorSigmaAlpha}{a}
\newcommand{\PriorSigmaBeta}{b}
\newcommand{\PriorBetaMean}{\mubf_{_{\bbeta}}}
\newcommand{\PriorBetaVar}{\Sigma_{_{\bbeta}}  }
\newcommand{\mvnormPdf}[4]{\frac{1}{ \left({2\pi}\right)^{\frac{#4}{2}} \sqrt{\vert{#3}\vert}}{\exp\left[ - \frac{1}{2}(#1- #2)^T {#3}^{-1} (#1- #2)\right]}      }
\newcommand{\InvGammaPdf}[3]{ \frac{(#1)^{-#2+1}}{\Gamma\left( #2\right) } \exp\left[ -\frac{{#3}}{{#1}} \right] }

 \newcommand{\byTilde}{\tilde{\by}}
 
 \newcommand{\TrfSigma}{\varsigma}
 \newcommand{ \Normal}{\text{Normal}}
 \newcommand{\GlobalPar}{\tau}
\newcommand{\LocalPar}{\psi}